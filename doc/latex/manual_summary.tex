\chapter{Introduction}
In the 1980's \cite{Lai83}, \cite{Tar84} and \cite{Mor87} suggested a new inversion strategy known as full waveform inversion (FWI). This method aims to reconstruct multi-parameter images of the subsurface by iteratively minimising the misfit between modeled and observed data. Thus it takes the full information content of the seismograms into account and can offer a significantly improved resolution compared to conventional methods. The optimisation problem is generally solved with gradient-based methods, which can be implemented very efficiently for FWI using the adjoint approach \citep[e.g.][]{Tar84, Mor87}. \\
Different approximations are generally used to limit the large number of unknown subsurface parameters and to mitigate the computational costs of the inversion. Many applications are performed in the 2D approximation. This leads to an enormous decrease of the number of subsurface parameters. Still, the 2D approximation is unable to explain 3D scattering and can lead to artefacts in 3D heterogeneous medium. Furthermore the use of a 3D FWI offers the possibility to invert for 3D structures and to gain a 3D image of the subsurface. Often, wave propagation in FWI applications is described with the (visco-) acoustic wave equation. Herby only compressional waves are considered which can be sufficient in marine seismics. However, in land seismics the abundance of shear waves and surface waves favours the (visco-) elastic wave equation. Still,  studies on the implementation and application of 3D elastic FWI as performed by \cite{Epa08}, \cite{Fic09},  \cite{Cas11}, \cite{Gua12} and \cite{But13} are rare and computationally expensive.\\
\\
IFOS3D is a 3D elastic full waveform inversion tool which aims to resolve the elastic parameters (compressional and shear wave velocity and density) of the 3D subsurface. It is based on the conjugate gradient method. For a good computational performance the gradient is calculated with a time-frequency approach. Hereby the wavefields are simulated with the fast and efficient finite-difference forward solver SOFI3D in time domain \citep{boh02}. The gradient calculation is then performed in frequency domain for discrete frequencies. A discrete Fourier transform on the fly enables the transformation from time to frequency domain. For an optimisation of the gradient method IFOS3D offers the calculation of a diagonal Hessian approximation and application of the L-BFGS method.\\
The IFOS3D program is an extension of the SOFI3D forward modeling code for inversion and is thus closely linked to this program. SOFI3D is based on the FD approach described by \cite{Vir86} and \cite{Lev88}. The present program SOFI3D (elastic version) has the following extensions
\begin{itemize}
\item employs higher order FD operators,
\item applies Perfectly Matched Layer boundary conditions at the edges of the numerical mesh \citep{Kom07},
\item works in MPI parallel environment ONLY, i.e. SOFI3D is a implementation based on a domain decomposition
\citep{boh02}.
\end{itemize}
The manual of SOFI3D gives a detailed description of the forward modelling solver, its features and the corresponding input parameters.\\
\\
IFOS3D was successfully tested in different synthetic applications including transmission and surface acquisition geometries \citep{But13,But15}. A detailed description of the theory and implementation of the program is given by \cite{But15}. This manual will give an overview about the structure and implementation of IFOS3D which is necessary to run the program. The different input parameters are decribed in detail. Additionally IFOS3D comes with a toy example which is described in this guide. It can be performed in a few steps and therefore offers a good access to IFOS3D.
 